\documentclass[12pt, letterpaper]{article}
\usepackage[margin=1in]{geometry}
\usepackage{amsmath}
\usepackage{tikz} % for timeline
\usepackage{titling} % to make title higher
\usepackage{enumitem}
\setlength{\droptitle}{-1.2in}

\title{\Large Project Proposal: Comparing ARIMA, RNN, and Transformer Models for Predicting Financial Market Crashes Using Market Data\\}
\author{Ping-Chieh Tu, Adrien Bélanger, Inigo Torres}
\date{November 3$^{\text{rd}}$, 2024}

\begin{document}

\maketitle

\begin{enumerate}[label=]

    \item \textbf{Background, Motivation, and Project Thesis}

    Predicting financial market crashes has long been an important problem due to the severe consequences of such events. For instance, the crash of 1929 preceded the Great Depression, the most severe economic crisis in history. Modern machine learning techniques can assist economists in predicting crashes and mitigating their effects.

    Our project focuses on using time series analysis to predict market crashes, defined as a decline of 20\% or more in a major market index from its recent peak within a short period. We will compare the performance of three widely used models for time-series forecasting: 
    - Autoregressive Integrated Moving Average (ARIMA),
    - Recurrent Neural Networks (RNN), including Long Short-Term Memory (LSTM), and 
    - Transformer-based architectures.

    The goal is to identify which model or combination of models performs best in predicting market crashes.

    \item \textbf{Objectives}

        \begin{itemize}
            \item[-] Conduct a literature review to identify best practices and previous implementations of ARIMA, RNNs, and Transformers for time series forecasting.
            \item[-] Implement these models and adapt them to the financial crash prediction context.
            \item[-] Train and test these models on historical financial data, including stock prices, trading volumes, and volatility indices.
            \item[-] Compare the models based on:
              \begin{enumerate}[label=\alph*.]
                \item Accuracy in crash prediction,
                \item Computational efficiency,
                \item Robustness across different market conditions.
              \end{enumerate}
            \item[-] Analyze the results to identify the most favorable approach and compare with findings from existing literature.
        \end{itemize}

    \item \textbf{Success Criteria}

        \begin{enumerate}[label=-]
            \item Successfully implement ARIMA, RNN, and Transformer models with appropriate hyperparameter tuning and preprocessing.
            \item Define and apply a clear evaluation metric, such as F1-score, precision, recall, or mean absolute error (MAE), tailored for time series data.
            \item Models must achieve performance comparable to or better than those reported in the literature.
            \item Provide a detailed empirical comparison that highlights the strengths and weaknesses of each model in different scenarios.
        \end{enumerate}

    \item \textbf{Methodology}

        To implement our project, we will use a variety of historical financial datasets, such as stock prices, trading volumes, and volatility indices. Our sources include:
        - Yahoo Finance for stock market data and historical prices,
        - Quandl for macroeconomic datasets,
        - The World Bank Open Data for global financial statistics.

        Additionally, we will tag historical crashes manually based on reports from the National Bureau of Economic Research. Sentiment analysis on news articles and social media metadata may also be incorporated to enhance the dataset.

        For implementation:
        - ARIMA**: Model hyperparameters \( p, d, q \) will be chosen using Autocorrelation Function (ACF) and Partial ACF.
        - RNNs**: We will focus on LSTM cells to mitigate the vanishing gradient problem.
        - Transformers: We will use self-attention mechanisms for handling long-term dependencies in time series data.

        We will use Python libraries such as \texttt{pandas} and \texttt{NumPy} for data preprocessing and \texttt{scikit-learn}, \texttt{TensorFlow}, and \texttt{PyTorch} for model development. Complex models will be trained on cloud platforms like Google Colab or AWS.

    \item \textbf{Evaluation Metrics and Analysis}

        Financial crashes are defined as a rapid decline of 20\% or more from a recent peak over a short period. We will use this definition to label market conditions in the dataset. Evaluation will focus on:
        - Precision and recall for identifying crash periods,
        - Mean Absolute Error (MAE) for time-series forecasting accuracy,
        - Computational runtime to assess efficiency.

        The models will be compared across these metrics under various market conditions to identify their strengths and weaknesses.

    \item \textbf{Project Plan and Timeline}
    
        \begin{enumerate}[label=-]
            \item Literature Review: Identify relevant datasets and models, review prior implementations, and gather expected performance benchmarks. (Deadline: November 8, 2024)
            \item Implementation: Develop and train ARIMA, RNN, and Transformer models with preprocessing and hyperparameter tuning. (Deadline: Mid-November)
            \item Analysis and Report: Compare model results, analyze findings, and draft the final report and presentation. (Deadline: End of November)
        \end{enumerate}
        
        \begin{tikzpicture}[every node/.style={align=center, font =\small}]
          \draw[->] (0,0) -- (14,0);
        
          \foreach \x/\y/\label in {
            2/{Literature\\Review}/8th of November,
            7/{Implementation of \\ models and training}/Mid November,
            12/{Analysis, Final Report \\ and Presentation}/End of November
          } {
            \fill (\x,0) circle (2pt);
            \draw (\x,0) node[above=5pt] {\y} node[below=5pt] {\label};
          }
        \end{tikzpicture}    

    
    \item \textbf{Conclusion}

    By implementing and comparing ARIMA, RNN, and Transformer models on historical financial data, this project will contribute to the understanding of machine learning applications in financial time series analysis. The findings will provide a comprehensive empirical review and practical guidance for future work in this domain.

\end{enumerate}

\end{document}
