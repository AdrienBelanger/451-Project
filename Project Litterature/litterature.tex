\documentclass[12pt, letterpaper]{article}
\usepackage[margin=1in]{geometry}
\usepackage{amsmath}
\usepackage{tikz} % for timeline
\usepackage{titling} % to make title higher
\usepackage{enumitem}
\usepackage{ragged2e} % justification
\usepackage{comment} % for commenting out multiple lines
\usepackage[
backend=biber,
style=ieee,
sorting=none,
]{biblatex}
\addbibresource{bibliography.bib}



\setlength{\droptitle}{-1.2in}
\pretitle{\vspace{0em}\begin{center}\Large} 
\posttitle{\par\end{center}\vspace{-1.25em}}
\preauthor{\vspace{0em}\begin{center}} 
\postauthor{\par\end{center}\vspace{-1.5em}} 
\predate{\vspace{0em}\begin{center}} 
\postdate{\par\end{center}\vspace{-0.5em}} 


\title{Empirical Review of Models used for Predicting Financial Market Crashes Using Market Data}
\author{\large Project: Literature Review \vspace{0.75em} \\ \normalsize By Ping-Chieh Tu, Adrien Bélanger, and Inigo Torres}
\date{November 22$^{\text{nd}}$ 2024}

\begin{document}

\maketitle 

\justifying % justifying our text instead of the annoying list like last time
\begin{comment}
Overall objective to keep in mind:

"In this milestone, the objective is to review the related literature to your proposal. This will better inform your methodology for your project if it involves a new idea, and it is necessary if you are comparing existing methods for a certain domain. It may even lead to a change of proposal, once you learn about existing methods out there. If you are producing a literature survey on a research topic, in this stage, you just provide a "breadth" review, in which you emphasize covering as many related works as possible and providing some preliminary organization without going into much detail."\\

Evaluation Criterias:

- Putting your proposal into context of related literature

- Coverage (are you adequately covering most relevant works)
\end{comment}
\subsection*{Background and Introduction \textcolor{red}{Adrien (Goes with Background)} }
We are not the first to approach this subject. 
Multiple papers have approached prediction of stock market crashes using Machine Learning. 
In a time series analysis approach, models used have traditionally included RNNs and Arima, and more recently Transformers \cite{Ahmed} \cite{ArunKumar}. 
These models have been successfully and often used in Market Crash prediction \cite{Okpeke}. 


Reviews and comparisons of these models, such as this project aims to do, have been made such as \cite{Okpeke}, but a comprehensive empirical review of Time-Series Analysis models on equal footing is lacking in literature. 
This project aims to address this lack, by providing an empirical comparison of the three aforementionned commonly used models in Time-Series Analysis (Arima models, RNNs and Transformers) to predict Market Crashes. \cite{Ahmed} \cite{ArunKumar}
We shall use data freely available on the Yahoo finance database, and tag historically factual Market Crashes by hand, as there are only few. We shall use the crashes as listed in this National Bureau of Economic Research's report. \cite{Mishkin}

%\subsection*{Background on Time Series and Financial Market Crashes \textcolor{red}{Adrien}}

% - Define time series in financial analysis -

% - Explain market crashes and justify our definition

%- Review work on financial crash prediction, review different methods

% - Identify gaps in literature that our project adresses and what is new about it

\subsection*{Methodology}
When first looking at a time series, we need to know that whether there it is \textbf{stationary} or \textbf{non-stationary}, which means 

- Justify why we chose those three models by finding similar work \textcolor{red}{Oscar ?} This might be good: https://oarjst.com/sites/default/files/OARJST-2024-0095.pdf
    \subsubsection*{ARIMA on predicting market crash \textcolor{red}{Oscar}}
    Time series in stock market are mostly non-stationary. But except for differences in the local level or in trend, all parts of the series behave in a similar way [Box et al. 2015]. Box et al suggested a model called Autoregressive Integrated Moving Average (ARIMA) model which combines autoregression model and moving average model. It is a models that can be used in non-stationary time series to make prediction [M K Ho et al 2021 J. Phys.: Conf. Ser. 1988 012041].
    %https://iopscience.iop.org/article/10.1088/1742-6596/1988/1/012041/pdf
    The ARIMA model ARIMA($p,d,q$) has general form [Box et al. 2015]:
    % http://repo.darmajaya.ac.id/4781/1/Time%20Series%20Analysis_%20Forecasting%20and%20Control%20%28%20PDFDrive%20%29.pdf
    \begin{align*}
        \varphi (B) z_t = \phi (B)\nabla^d z_t = \theta_0 + \theta (B)a_t
    \end{align*}
    where
    \begin{enumerate}[label=\arabic*.]
        \item $\phi(B)$ is the autoregressive part with $p$ degrees,
        \item $\theta (B)$ is the moving average part with $q$ degrees,
        \item $\nabla$ is the integrated (degree of differencing) part with $d$ degrees,
        \item and $\theta_0$ is the constant term.
    \end{enumerate}
    Using ACF (autocorrelation function) and PACF (partial ACF) can help us decide the hyperparameters $p, d, q$ for building our model [Hyndman et al. 2018].
    % https://otexts.com/fpp2/
    \subsubsection*{Reccurent Neural Networks \textcolor{red}{Adrien}}
    %- Overview of Recurrent Neural Networks (RNNs)
    Recurrent Neural Networks are a class of neural network architectures designed to detect patterns in sequential data, such as handwriting, genomes, text, or numerical time series. In time-series, they are used to make sequential predictions based on sequential inputs. \cite{Schmidt} In our context, they usually take the form of Long Term Short Memory (LSTM) cells \cite{Hansika}, which we will use in our project, to mitigate the vanishing gradient problem. They do so by having cells with three gates: An input, output and a forget gate combined with a feedback loop to enhance long-term accuracy. These gates help control how the information flows through the network. \cite{Hansika} They have been used in multiple projects accounting to market Crashes, such as this project led by Tolo et al. \cite{Tolo} It is not well established when RNNs and Arimas outperform one another \cite{Hansika}, this project will take a step into clearing up the most appropriate in the context of Market Crashes prediction.

    %- Review its application in time series analysis in our context
    \subsubsection*{Transformers \textcolor{red}{Inigo}}
Transformers are neural network architectures that are based on "self-attention mechanisms" (allowing the model to weigh the importance of different elements in the input by computing attention scores between all positions) to model dependencies in sequential data. while Transofrmers were originally developped for natural language processing tasks like machine translation [Vaswani et al., "Attention is All You Need"], they have become popular for time series forecasting due to their ability to handle long-range dependencies.

Transformers can capture complex temporal patterns by assigning varying importance to different time steps, which is particularly useful in fluctuative/volatile financial markets. In fact, Zhou et al. demonstrated this with the Informer model, which efficiently handles long sequences and improves forecasting accuracy using self-attention [Zhou et al., "Informer: Beyond Efficient Transformer for Long Sequence Time-Series Forecasting"]. Moreover, Lim et al. were even able to outperform traditional methods by combining Transformer architecture with recurrent layers [Lim et al., "Temporal Fusion Transformers for Interpretable Multi-horizon Time Series Forecasting"].

Hence, the literature suggest that Tranformers are a pertinent choice for identifying complex patterns in finatial time series, as they can model dependencies accros mutiple time sclaes, potentially improving the precision of crash predictions compared to traditional methods like ARIMA and standard RNNs



\subsection*{Criterias and Analysis \textcolor{red}{Oscar?}}
- Summarize criterias and metrics in literature for comparing models.

- Justify the selection of those comparison methods based on sources

\pagebreak
\printbibliography

% Modify proposal in proposal.tex
%\subsection*{Adapt proposal: \textcolor{red}{Inigo}}
\end{document}