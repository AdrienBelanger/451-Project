\documentclass[12pt, letterpaper]{article}
\usepackage[margin=1in]{geometry}
\usepackage{amsmath}
\usepackage{tikz} % for timeline
\usepackage{titling} % to make title higher
\usepackage{enumitem}
\usepackage{ragged2e} % justification

\setlength{\droptitle}{-1.2in}
\pretitle{\vspace{0em}\begin{center}\Large} 
\posttitle{\par\end{center}\vspace{-1.25em}}
\preauthor{\vspace{0em}\begin{center}} 
\postauthor{\par\end{center}\vspace{-1.5em}} 
\predate{\vspace{0em}\begin{center}} 
\postdate{\par\end{center}\vspace{-0.5em}} 


\title{Empirical Review of Models used for Predicting Financial Market Crashes Using Market Data}
\author{\large Project: Literature Review \vspace{0.75em} \\ \normalsize By Ping-Chieh Tu, Adrien Bélanger, and Inigo Torres}
\date{November 22$^{\text{nd}}$ 2024}

\begin{document}

\maketitle 

\justifying % justifying our text instead of the annoying list like last time

Overall objective to keep in mind:

"In this milestone, the objective is to review the related literature to your proposal. This will better inform your methodology for your project if it involves a new idea, and it is necessary if you are comparing existing methods for a certain domain. It may even lead to a change of proposal, once you learn about existing methods out there. If you are producing a literature survey on a research topic, in this stage, you just provide a "breadth" review, in which you emphasize covering as many related works as possible and providing some preliminary organization without going into much detail."\\

Evaluation Criterias:

- Putting your proposal into context of related literature

- Coverage (are you adequately covering most relevant works)

\subsection*{Introduction}
"Start with a broad picture of the research area in order to position your work. It is important to demonstrate perspective."

\subsection*{Background on Time Series and Financial Market Crashes}

- Define time series in financial analysis

- Explain market crashes and justify our definition

- Review work on financial crash prediction, review different methods

- Identify gaps in literature that our project adresses and what is new about it

\subsection*{Review of Relevant Models and Methods}
- Justify why we chose those three models by finding similar work
    \subsubsection*{Arima}
    - Overview of ARIMA 

    - Review its application in time series analysis in our context
    \subsubsection*{Reccurent Neural Networks}
    - Overview of Recurrent Neural Networks (RNNs)

    - Review its application in time series analysis in our context
    \subsubsection*{Transformers}
    - Overview of Transformers

    - Review its application in time series analysis in our context


\subsection*{Criterias and Analysis}
- Summarize criterias and metrics in literature for comparing models.

- Justify the selection of those comparison methods based on sources


\end{document}
