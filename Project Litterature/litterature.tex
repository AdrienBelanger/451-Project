\documentclass[12pt, letterpaper]{article}
\usepackage[margin=1in]{geometry}
\usepackage{amsmath}
\usepackage{tikz} % for timeline
\usepackage{titling} % to make title higher
\usepackage{enumitem}
\usepackage{ragged2e} % justification
\usepackage{comment} % for commenting out multiple lines
\setlength{\droptitle}{-1.2in}
\pretitle{\vspace{0em}\begin{center}\Large} 
\posttitle{\par\end{center}\vspace{-1.25em}}
\preauthor{\vspace{0em}\begin{center}} 
\postauthor{\par\end{center}\vspace{-1.5em}} 
\predate{\vspace{0em}\begin{center}} 
\postdate{\par\end{center}\vspace{-0.5em}} 


\title{Empirical Review of Models used for Predicting Financial Market Crashes Using Market Data}
\author{\large Project: Literature Review \vspace{0.75em} \\ \normalsize By Ping-Chieh Tu, Adrien Bélanger, and Inigo Torres}
\date{November 22$^{\text{nd}}$ 2024}

\begin{document}

\maketitle 

\justifying % justifying our text instead of the annoying list like last time
\begin{comment}
Overall objective to keep in mind:

"In this milestone, the objective is to review the related literature to your proposal. This will better inform your methodology for your project if it involves a new idea, and it is necessary if you are comparing existing methods for a certain domain. It may even lead to a change of proposal, once you learn about existing methods out there. If you are producing a literature survey on a research topic, in this stage, you just provide a "breadth" review, in which you emphasize covering as many related works as possible and providing some preliminary organization without going into much detail."\\

Evaluation Criterias:

- Putting your proposal into context of related literature

- Coverage (are you adequately covering most relevant works)
\end{comment}
\subsection*{Background and Introduction \textcolor{red}{Adrien (Goes with Background)} }
We are not the first to approach this subject. 
Multiple papers have approached prediction of stock market crashes using Machine Learning. 
Some have used Support Vector Machines, random forests, and others Neural Networks [Okpeke, Predicting Stock Market]. 
In a time series analysis approach, models used have traditionally included RNNs and Arima, and more recently Transformers. [Sabeen Ahmed, Transformers ... ] [Arunkumar, Comparative Analysis]. 
These models have been used in Market Crash prediction [Okpeke]. 
Others have tried using diverse databases and models such as Social Media interactions [Chhajer, The Applications.. ] using Natural Language Processing. As this is a very economically valuable foresight, a lot of research has been done on the subject.


Reviews and comparisons of these models, such as this project aims to do, have been made such as [Okpeke, Predicting], but a comprehensive empirical review of Time-Series Analysis models on equal footing is lacking in litterature. 
This project aims to address this lack, by providing an empirical comparison of three commonly used models in Time-Series Analysis (Arima models, RNNs and Transformers) to predict Market Crashes. [Sabeen Ahmed, Transformers ... ] [Arunkumar, Comparative Analysis]. 
We shall use data freely available on the Yahoo finance database, and tag historically factual Market Crashes by hand, as there are only few. We shall use the crashes as listed in []. This methodology (or closely related) is common procedure, and has been used for this in the past [][]. Specific Criterias for comparison and analysis shall be discussed in a further section.

%\subsection*{Background on Time Series and Financial Market Crashes \textcolor{red}{Adrien}}

% - Define time series in financial analysis -


% - Explain market crashes and justify our definition

%- Review work on financial crash prediction, review different methods

% - Identify gaps in literature that our project adresses and what is new about it

\subsection*{Methodology}
- Justify why we chose those three models by finding similar work \textcolor{red}{Oscar ?} This might be good: https://oarjst.com/sites/default/files/OARJST-2024-0095.pdf
    \subsubsection*{Arima \textcolor{red}{Oscar}}
    - Overview of ARIMA 

    - Review its application in time series analysis in our context
    \subsubsection*{Reccurent Neural Networks \textcolor{red}{Adrien}}
    %- Overview of Recurrent Neural Networks (RNNs)
    Recurrent Neural Networks are a class of neural network architectures designed to detect patterns in sequential data, such as handwriting, genomes, text, or numerical time series. [Schmidt, Recurrent Neural Networks...]. They have been used in multiple projects accounting to market Crashes, such as [] and [].

    %- Review its application in time series analysis in our context
    \subsubsection*{Transformers \textcolor{red}{Inigo}}
Transformers are neural network architectures that are based on "self-attention mechanisms" (allowing the model to weigh the importance of different elements in the input by computing attention scores between all positions) to model dependencies in sequential data. while Transofrmers were originally developped for natural language processing tasks like machine translation [Vaswani et al., "Attention is All You Need"], they have become popular for time series forecasting due to their ability to handle long-range dependencies.

Transformers can capture complex temporal patterns by assigning varying importance to different time steps, which is particularly useful in fluctuative/volatile financial markets. In fact, Zhou et al. demonstrated this with the Informer model, which efficiently handles long sequences and improves forecasting accuracy using self-attention [Zhou et al., "Informer: Beyond Efficient Transformer for Long Sequence Time-Series Forecasting"]. Moreover, Lim et al.  were evn able to outperform traditional methods by combining Transformer architecture with recurrent layers [Lim et al., "Temporal Fusion Transformers for Interpretable Multi-horizon Time Series Forecasting"].

Hence, the litteratrue suggest that Transformers are a pertinent choice for identifying complex patterns in finatial time series, as they can model dependencies accros mutiple time sclaes, potentially improving the precision of crash predictions compared to traditional methods like ARIMA and standard RNNs



\subsection*{Criterias and Analysis \textcolor{red}{Oscar?}}
- Summarize criterias and metrics in literature for comparing models.

- Justify the selection of those comparison methods based on sources



\subsection*{Adapt proposal: \textcolor{red}{Inigo}}
\end{document}