\documentclass[12pt, letterpaper]{article}
\usepackage[margin=1in]{geometry}
\usepackage{amsmath}
\usepackage{tikz} % for timeline
\usepackage{titling} % to make title higher

\setlength{\droptitle}{-1in}
\title{
  Project Proposal \\
  \Large Comp 451}

\author{Ping-Chieh Tu, Adrien Bélanger, Inigo ??}
\date{November 3$^{\text{rd}}$ 2024}

\begin{document}

\maketitle 
% No space for the section names, its basically a text with paragraphs so no need

\begin{enumerate}
    \item Background, motivation and Project Thesis (Oscar and All)
    
    This project compares several related models and approaches to predict market crashes using market data.
    
    \item Objectives and Success Criteria (Adrien)
    
        Objectives (will write them more thoroughly):
        \begin{itemize}
            \item [-] Find and correctly implement commonly used models for Time Series Analysis
            \item[-] Run and Analyze Results on the same dataset and objective
            \item[-] Compare the results of each model and discuss their strengths and weaknesses
            \item[-]  Conclude the most favorable approach indicated by our results and compare with literature
        \end{itemize} 
        Success Criteria
        \begin{itemize}
            \item [-] Placeholder
        \end{itemize}
        
    \item Project Plan and timeline (Adrien and All)
        
        Project plan (Draft)
        \begin{itemize}
            \item [-] Literature Review
            \item [-] Implementation of models and training
            \item [-] Analysis, Final Report and presentation work
        \end{itemize}
    
        Timeline
        
        \begin{tikzpicture}[every node/.style={align=center, font =\small}]
          \draw[->] (0,0) -- (14,0);
        
          \foreach \x/\y/\label in {
            2/{Literature\\Review}/8th of November,
            7/{Implementation of \\ models and training}/Mid November,
            12/{Analysis, Final Report \\ and Presentation}/End of November
          } {
            \draw (\x,0) node[above=5pt] {\y} node[below=5pt] {\label};
          }
        \end{tikzpicture}    
        
    \item Discussing some datasets and how we will implement our project (platform, etc) (Inigo)





    
\end{enumerate}


\end{document}
